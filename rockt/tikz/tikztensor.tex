%\NeedsTeXFormat{LaTeX2e}[1994/06/01]
%\ProvidesPackage{tikztensor}[2014/07/25 tikztensor]
%\RequirePackage{tikz}

\usepackage{tikz}
\usetikzlibrary{calc,trees,positioning,arrows,chains,shapes.geometric,%
  decorations.pathreplacing,decorations.pathmorphing,shapes,%
  matrix,shapes.symbols,fit,decorations}

\usepackage{xcolor}
\usepackage{xparse}
%\usetikzlibrary{matrix, calc, arrows, decorations.markings}

% relative positioning
\newcommand{\at}[3]{
  \begin{scope}[shift={(#1,#2)}]
    #3
  \end{scope} 
}

% scaling
\newcommand{\scale}[2]{
  % \begin{scope}[transform canvas={scale=#1}]
  \begin{scope}[scale=#1]
    #2
  \end{scope}
}

% naming
\newcommand{\name}[2]{
  \begin{scope}[local bounding box=#1]
    #2
  \end{scope}
}

% shorting both ends of an arrow
\tikzset{ shorten <>/.style={ shorten >=#1, shorten <=#1 } }

% thick arrow
\tikzstyle{arrow}=[draw, -latex, very thick] 

%% document
\newcommand{\doc}{
  \node[draw, thick, minimum width=1cm, minimum height=1.4cm, fill=white] at (0,0) {};
  \foreach \y in {-.5, -.3, -.1, .1, .3, .5} {
    \draw[thick] (-.4, \y) -- (.4, \y);
  }
}

% corpus
\newcommand{\corpus}{
  \foreach \x in {.2, .1, 0} {
    \at{\x}{\x}{\doc};
  }
}

% database
\tikzset{%
  database/.style={
    cylinder,
    cylinder uses custom fill,
    cylinder body fill=black!10,
    cylinder end fill=black!10,
    shape border rotate=90,
    aspect=0.25,
    thick,
    draw
  }
}

% user
\newcommand{\user}{
  \node (0) at (0, 0.5) {};
  \node (1) at (0, -0.5) {};
  \node (2) at (-0.5, -1.25) {};
  \node (3) at (0.5, -1.25) {};
  \node (4) at (0.75, 0) {};
  \node (5) at (-0.75, 0) {};
  \node (6) at (0, 1) {};
  \draw[thick] (0.center) to (1.center);
  \draw[thick] (1.center) to (2.center);
  \draw[thick] (1.center) to (3.center);
  \draw[thick] (5.center) to (4.center);
  \draw[thick] (0,.9) circle (.4);
}

\let\auser\user  
%\usepackage{xcolor}
%general
\colorlet{dark-blue}{blue!50!black}
\colorlet{dark-cyan}{cyan!75!black}
\colorlet{dark-purple}{purple!50!black}
\colorlet{dark-red}{red!75!black}
\colorlet{dark-green}{green!75!black}
\colorlet{dark-orange}{orange!50!black}
\colorlet{dark-gray}{black!75}
\colorlet{light-gray}{black!30}
%named
\colorlet{hidden}{light-gray}
%package-specific
\colorlet{todo}{red!85!black}
\colorlet{todoref}{purple!70!black}
%UCL
\definecolor{ucl-purple}{RGB}{80,7,120}
\definecolor{ucl-navy-blue}{RGB}{0,40,85}
\definecolor{ucl-mid-green}{RGB}{143,153,62}
\definecolor{ucl-burgundy}{RGB}{147,39,44}
\definecolor{ucl-mid-red}{RGB}{224,60,49}
\definecolor{ucl-orange}{RGB}{234,118,0}
\definecolor{ucl-yellow}{RGB}{246,190,0}
% short-cuts
\colorlet{ucl-blue}{ucl-navy-blue}
\colorlet{ucl-green}{ucl-mid-green}
\colorlet{ucl-red}{ucl-burgundy}

  


\newcommand{\tikztensorx}{1}
\newcommand{\tikztensory}{1}
\newcommand{\tikztensorz}{1}

\newcommand{\setcoordinates}[3]{
  \renewcommand{\tikztensorx}{#1}
  \renewcommand{\tikztensory}{#2}
  \renewcommand{\tikztensorz}{#3}

  \coordinate (0) at (0,0,0);
  \coordinate (x) at (\tikztensorx,0,0);
  \coordinate (y) at (0,\tikztensory,0);
  \coordinate (z) at (0,0,-\tikztensorz);
  \coordinate (xy) at (\tikztensorx,\tikztensory,0);
  \coordinate (xz) at (\tikztensorx,0,-\tikztensorz);
  \coordinate (yz) at (0,\tikztensory,-\tikztensorz);
  \coordinate (xyz) at (\tikztensorx,\tikztensory,-\tikztensorz); 
}

\newcommand{\debugcoordinates}{
  \foreach \xy in {0, x, y, z, xy, xz, yz, xyz}{
    \node at (\xy) {\xy};
  }
}


\newcommand{\tensorback}[1]{
  \name{bottom}{\draw[black, fill = black!5, #1] (0) -- (x) -- (xz) -- (z) -- cycle;}
  \name{back}{\draw[black, fill = black!5, #1] (z) -- (xz) -- (xyz) -- (yz) -- cycle;}
  \name{left}{\draw[black, fill = black!5, #1] (0) -- (z) -- (yz) -- (y) -- cycle;}
}

\newcommand{\tensorfront}[1]{
  % right
  \draw[black, fill = black, fill opacity=0.15, #1] (x) -- (xz) -- (xyz) -- (xy) -- cycle;
  % front
  \draw[black, fill = black, fill opacity=0.05, #1] (0) -- (x) -- (xy) -- (y) -- cycle;
  % top
  \draw[black, fill = black, fill opacity=0.1, #1] (y) -- (xy) -- (xyz) -- (yz) -- cycle;
}

\DeclareDocumentCommand{\tensor}{O{} O{} m m m}{
  \name{lhs}{
   % \setcoordinates{#3 * 0.25}{#4 * 0.25}{#5 * 0.25}
   % \tensorback{#1}
    \tensorgrid{#3}{#4}{#5}
    #2
    
    \setcoordinates{#3 * 0.25}{#4 * 0.25}{#5 * 0.25}
    \tensorfront{#1}
  }
}



\newcommand{\tensorgrid}[3]{
    \foreach \i in {0,...,#1} {
      \draw[opacity=0.2] (\i*0.25,0) -- (\i*0.25,#2*0.25);
      \draw[opacity=0.2] (\i*0.25,#2*0.25) -- (\i*0.25,#2*0.25,-#3*0.25); 
    }
    \foreach \i in {0,...,#2} {
      \draw[opacity=0.2] (0,\i*0.25) -- (#1*0.25,\i*0.25);
      \draw[opacity=0.2] (#1*0.25,\i*0.25) -- (#1*0.25,\i*0.25,-#3*0.25); 
    }
    \foreach \i in {0,...,#3} {
      \draw[opacity=0.2] (0,#2*0.25,-\i*0.25) -- (#1*0.25,#2*0.25,-\i*0.25);
      \draw[opacity=0.2] (#1*0.25,0,-\i*0.25) -- (#1*0.25,#2*0.25,-\i*0.25);
    }
  % \pgfmathsetmacro\iIx{#1 -1}
  % \pgfmathsetmacro\jIx{#2 -1}

  % \ifnum#3=0
  % \pgfmathsetmacro\kIx{#3}
  % \pgfmathsetmacro\cellDepth{0}
  % \else
  % \pgfmathsetmacro\kIx{#3 -1}
  % \pgfmathsetmacro\cellDepth{1}
  % \fi   

  % \foreach \k in {\kIx,...,0} {
  %   \pgfmathsetmacro\cellOpacity{0.01 * \k}
  %   \foreach \i in {0,...,\iIx} {
  %     \foreach \j in {0,...,\jIx} {
  %       \tensorshift[\i][\j][\k]{\tensor[draw=black, fill=black, opacity=\cellOpacity]{1}{1}{\cellDepth}}
  %     }
  %   }
  % }
}


\DeclareDocumentCommand{\tensorop}{O{.5em} O{1.5em} m}{
  \node[right = #1 of lhs.south east, yshift=#2] (op) {#3};
  \node[right = #1 of 0 -| op] {};
  \pgfgetlastxy{\nextx}{\nexty};
}

\DeclareDocumentCommand{\calculatenext}{O{.5em} O{2em} m}{
  \node[right = #1 of #3.south east, yshift=#2] (tmp) {};
  \pgfgetlastxy{\nextx}{\nexty};
}

\DeclareDocumentCommand{\setnext}{O{} m}{
  \node[#1] at (#2) {};
  \pgfgetlastxy{\nextx}{\nexty};
}

\DeclareDocumentCommand{\labelaxis}{O{black} m m m m}{
  \draw[ultra thick, #1] (#2) -- node[midway, #4] {#5} (#3);
}

\DeclareDocumentCommand{\labelx}{O{blue} m}{
  \labelaxis[#1]{0}{x}{below}{#2}
}

\DeclareDocumentCommand{\labely}{O{red} m}{
  \labelaxis[#1]{0}{y}{left}{#2}
}

\DeclareDocumentCommand{\labelz}{O{dark-green} m}{
  \labelaxis[#1]{y}{yz}{above left}{#2}
}


\DeclareDocumentCommand{\tensorshift}{O{0} O{0} O{0} m}{
  \begin{scope}[shift={(#1 * 0.25,#2 * 0.25,-#3 * 0.25)}]
    #4
  \end{scope} 
}



%\endinput
